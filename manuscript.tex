\documentclass[12pt]{article}

% ------------------------------------------------
% 1. Packages
% ------------------------------------------------
\usepackage[utf8]{inputenc}
\usepackage{geometry}
\usepackage{xcolor}
\usepackage{etoolbox} % Döngü (loop) işlemleri için şart
\usepackage{clipboard}

% ------------------------------------------------
% 2. Clipboard Kurulumu
% ------------------------------------------------
\newclipboard{main_round2} % main_round2.cpy dosyasını oluşturur

% ------------------------------------------------
% 3. The "Smart" Revision Macro (Tekrar Önleyici)
% ------------------------------------------------
% Kullanım: \revised{1.1} veya \revised{1.2a, 1.2b} veya \revised{2.1, 3.4}
\newcommand{\revised}[2]{%
    % A. GÖRSEL KISIM:
    % Metni PDF'e sadece 1 kez basar.
    \textcolor{blue}{#2 (Reviewers #1)}%
    %
    % B. MANTIKSAL KISIM (Arka Plan):
    % ID listesindeki her eleman için döner, veriyi hafızaya alır.
    \renewcommand{\do}[1]{%
        \label{rev:##1}% Sayfa/Section referansı için etiket
        % "Phantom Box" Tekniği:
        % Kopyalama işlemini görünmez bir kutunun içinde yapıyoruz.
        % Böylece \Copy komutu ekrana tekrar çıktı vermez.
        \setbox0=\vbox{%
            \Copy{text:##1}{%
                \textcolor{blue}{#2 (Reviewers #1)}%
            }%
        }%
    }%
    \docsvlist{#1}% Döngüyü çalıştır
}

\title{Conference Simulation: Revised Manuscript}
\author{Huseyin Karaca}
\date{\today}

\begin{document}
\maketitle

\section{Introduction}
% SENARYO 1: Basit Revizyon (Tek bir yer)
Machine learning models often require careful tuning. 
\revised{1.1}{We have clarified the specific contribution of our hierarchical ensemble method in the abstract to address the ambiguity.}

\section{Methodology}
% SENARYO 2: "Dağınık" Revizyon (Reviewer 1.2 için hem burayı hem Sonucu değiştirdik)
In this section, we define the core parameters.
\revised{1.2a}{The parameter $\alpha$ represents the learning rate, which is adaptive.}

\subsection{Parameter Selection}
% SENARYO 3: Ortak İstek (Reviewer 2.1 ve Reviewer 3.1 aynı şeyi istedi)
The selection criteria are crucial.
\revised{2.1, 3.1}{We utilized a 5-fold cross-validation strategy to determine the optimal sparsity threshold.}
This addresses the concerns raised by multiple reviewers regarding overfitting.

\section{Conclusion}
The proposed method shows significant improvements.
% Reviewer 1.2'nin isteği üzerine yapılan ikinci değişiklik (1.2b)
\revised{1.2b}{Future work will focus on adaptive learning rates ($\alpha$) as discussed in the Methodology.}

\end{document}