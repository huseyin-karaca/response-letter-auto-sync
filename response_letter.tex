\documentclass[11pt]{article}

% ------------------------------------------------
% 1. Packages
% ------------------------------------------------
\usepackage[utf8]{inputenc}
\usepackage[T1]{fontenc}
\usepackage{geometry}
\usepackage{xcolor}
\usepackage{amsmath}
\usepackage[most]{tcolorbox}
\usepackage{xr}         % Referans okuma
\usepackage{clipboard}  % Metin okuma
\usepackage{etoolbox}   % Döngü işlemleri

% ------------------------------------------------
% 2. Manuscript Connection
% ------------------------------------------------
\externaldocument{main_round2} % Sayfa numaralarını al
\openclipboard{main_round2}    % Metin içeriğini al (.cpy dosyasından)

% ------------------------------------------------
% 3. Color Palette (SENİN SEÇİMLERİN)
% ------------------------------------------------
% Ana Kutu (ReviewCycle)
\definecolor{boxframe}{RGB}{40, 60, 100}     
\definecolor{boxfill}{RGB}{250, 250, 252}    

% Reviewer (Kiremit/Gri)
\definecolor{revbg}{RGB}{250, 240, 240}      
\definecolor{revtitle}{RGB}{180, 50, 50}     

% Author (Senin istediğin Gri Tonu)
\definecolor{authbg}{RGB}{235, 235, 235}     
\definecolor{authtitle}{RGB}{0, 80, 120}     

% Revision (Alice Blue - Değişmedi)
\definecolor{revisionbg}{RGB}{240, 248, 255} 

% ------------------------------------------------
% 4. Environments & Macros
% ------------------------------------------------

% A. Ana Çerçeve (ReviewCycle)
\newtcolorbox{ReviewCycle}[1]{
    enhanced, breakable, width=\textwidth,
    colframe=boxframe, colback=boxfill,
    coltitle=white, fonttitle=\bfseries\sffamily,
    title={#1}, arc=3pt, boxrule=0.8pt,
    top=5pt, bottom=5pt, left=5pt, right=5pt,
    drop fuzzy shadow,
    before upper={\parskip 0.5em}
}

% B. Reviewer Kutusu
\newcommand{\ReviewerText}[1]{%
    \begin{tcolorbox}[
        enhanced, frame hidden, colback=revbg,
        left=5pt, right=5pt, top=5pt, bottom=5pt, arc=2pt
    ]
        \textbf{\textcolor{revtitle}{Comment:}} 
        \itshape \textcolor{black!80}{#1}
    \end{tcolorbox}
    \vspace{0.05cm}
}

% C. Author Kutusu
\newcommand{\AuthorResponse}[1]{%
    \begin{tcolorbox}[
        enhanced, frame hidden, colback=authbg,
        left=5pt, right=5pt, top=5pt, bottom=5pt, arc=2pt
    ]
        \textbf{\textcolor{authtitle}{Response:}} 
        \normalfont \textcolor{black}{#1}
    \end{tcolorbox}
    \vspace{0.1cm}
}

% D. Akıllı Revizyon Kutusu (Çoklu ID Destekli)
\newcommand{\AutoRevision}[1]{%
    \begin{tcolorbox}[
        enhanced, breakable,
        colback=revisionbg, frame hidden,
        borderline west={3pt}{0pt}{blue!40}, % Mavi çizgi
        left=5pt, right=5pt, top=5pt, bottom=5pt,
        arc=2pt
    ]
        % Başlık
        \textbf{\textcolor{blue!40!black}{Revisions in Manuscript:}}
        
        % ID Listesi üzerinde döngü (örn: 1.2a, 1.2b)
        \renewcommand{\do}[1]{%
            \par\vspace{0.3cm} % Maddeler arası boşluk
            
            % 1. Konum Bilgisi (Section X, Page Y)
            {\small\bfseries\color{blue!60!black} $\hookrightarrow$ Section \ref{rev:##1}, Page \pageref{rev:##1}:}
            
            % 2. Metin (Alt satıra yapıştır)
            \par\nopagebreak
            \begingroup
            \itshape\small ``\Paste{text:##1}''
            \endgroup
        }
        \docsvlist{#1} % Listeyi işle
    \end{tcolorbox}
}

% ------------------------------------------------
% 5. Document Content
% ------------------------------------------------
\begin{document}

\begin{center}
    {\Large \textbf{Response to Reviewers}}\\[0.5em]
    \textbf{Project:} Conference Simulation Homework
\end{center}
\hrule \vspace{1em}

% --- SENARYO 1: Basit (1'e 1) ---
\begin{ReviewCycle}{Reviewer 1, Comment 1}
    \ReviewerText{The abstract is too vague. Please clarify the contribution.}
    \AuthorResponse{We have updated the abstract to explicitly state our contribution regarding the ensemble method.}
    \AutoRevision{1.1}
\end{ReviewCycle}

% --- SENARYO 2: Dağınık (Bir yorum, iki farklı yerdeki değişiklik) ---
\begin{ReviewCycle}{Reviewer 1, Comment 2}
    \ReviewerText{You mention 'alpha' but never define it consistently. Fix this in Intro and Conclusion.}
    \AuthorResponse{We have added the definition of alpha in the Methodology and updated the Conclusion to match.}
    % İki ID'yi virgülle çağırıyoruz
    \AutoRevision{1.2a, 1.2b}
\end{ReviewCycle}

% --- SENARYO 3: Ortak (İki yorum, tek bir ortak değişiklik) ---
% Reviewer 2 için cevap:
\begin{ReviewCycle}{Reviewer 2, Comment 1}
    \ReviewerText{How did you handle overfitting?}
    \AuthorResponse{We used 5-fold cross-validation.}
    \AutoRevision{2.1}
\end{ReviewCycle}

% Reviewer 3 için cevap (Aynı değişiklik referans veriliyor):
\begin{ReviewCycle}{Reviewer 3, Comment 1}
    \ReviewerText{Please specify the validation strategy.}
    \AuthorResponse{As suggested, we clarified the cross-validation strategy.}
    \AutoRevision{3.1}
\end{ReviewCycle}

\end{document}